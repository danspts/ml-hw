
\documentclass[12pt]{article}
\usepackage{amsmath}
\usepackage{amssymb}
\usepackage{xcolor}

\newcommand\tab[1][1cm]{\hspace*{#1}}

\linespread{1.3}


\title{Machine Learning from Data -IDC\\HW5–Theory+ SVM}
\author{227367455 and 323081950}

\begin{document}

\maketitle

\section*{\centering{Question 1)}}

\subsection*{a)}

Let $K,L$ be two kernels (operating on the same space)and let $ \alpha, \beta $ be two positive scalars \\
\tab Prove that $\alpha K + \beta L $ is a kernel.\\

Let $\langle x,y \rangle_1 = \alpha K(x,y) + \beta L(x,y) $ \\
We will show that $ \langle x,y \rangle_1 $ is a inner product by proving the four properties required to be an inner product operation. \\

\tab Symmetry:

$ \langle x,y \rangle_1 = \alpha K(x,y) + \beta L(x,y) $ \\
$ = \alpha K(y,x) + \beta L(y,x) $ by symmetry of inner products of $K$,$L$ \\
$ = \langle y,x \rangle_1 $
		
\tab Additivity:

$ \langle u_1 + u_2, v \rangle_1 = \alpha K(u_1 + u_2, v) + \beta L(u_1 + u_2, v) $ \\
$ = \alpha (K(u_1, v) + K(u_2, v)) + \beta (L(u_1, v) + L(u_2, v)) $ as $K$,$L$ are inner products \\
$ = (\alpha K(u_1, v) + \beta L(u_1, v)) + (\alpha K(u_2, v) + \beta L(u_2, v)) $ \\
$ = \langle u_1, v \rangle_1 + \langle u_2, v \rangle_1 $

\tab Homogeneity

$ \langle c\cdot u, v \rangle_1 = \alpha K(c\cdot u, v) + \beta L(c\cdot u, v) $ \\
$ = \alpha \cdot c \cdot K(u, v) + \beta \cdot c \cdot L(u, v) $ as $K$,$L$ are inner products \\
$ = c\cdot (\alpha K(u, v) + \beta L(u, v))  = c \cdot \langle u, v \rangle_1 $

\tab Positivity

$ \langle u, u \rangle_1 = \alpha K(u, u) + \beta L(u, u) $ \\
As $K$, $L$ are inner products, then $K(u, u) \geq 0$ and $L(u, u) \geq 0$ \\
As $\alpha,\beta > 0$ by the given assumption, then we can conclude that $ \langle u, u \rangle_1 \geq 0 $ as the right side is a sum and product of positive  numbers 

Now to show that $ \langle u, u \rangle_1 = 0 $ iff $ u = 0 $. \\
First assume that $ \langle u, u \rangle_1 = 0 $. \\
Then, as $\alpha,\beta > 0$, the only way this can occur are if $ K(u, u) = L(u, u) = 0 $. \\ 
As $K$, $L$ are inner products, this means that $ u = 0$. \\
Now assume that $ u = 0 $. \\
Notice that $ \langle u, u \rangle_1 = \alpha K(u, u) + \beta L(u, u) $. \\
As $K$, $L$ are inner products: \\
$ \langle u, u \rangle_1 = \alpha \cdot 0 + \beta \cdot 0 $ \\
$ = 0 $ which proves the other direction.
		
\subsection*{b)}

Provide (two different) examples of non-zero kernels $K,L$ (operating on the same space), so that:

\paragraph{\tab i.}

$K - L$ is a kernel

Let $K(x,y) = \varphi_1(x)\cdot\varphi_1(y)$ and $L(x,y) = \varphi_2(x)\cdot\varphi_2(y)$ \\
			Define $\varphi_1(x) = (2x, -4x^2)$ and $\varphi_2(x) = (x, x^2) $
			Notice that $ K(x,y) - L(x,y) = \varphi_1(x)\cdot\varphi_1(y) - \varphi_2(x)\cdot\varphi_2(y) $ \\
			$ = (2x, -4x^2)\cdot(2y, -4y^2) - (x, x^2)\cdot(y,y^2) $ \\
			$ = 2x \cdot 2y + (-4x^2) \cdot (-4y^2) - x \cdot y - x^2 \cdot y^2 $ \\
			$ = 4(x \cdot y) + 16(x^2 \cdot y^2) - x \cdot y - x^2 \cdot y^2 $ \\
			$ = 3(x \cdot y) + 15(x^2 \cdot y^2) $ \\
			$ = 3x \cdot 3y + 15x^2 \cdot 15y^2 $ \\
			This is a kernel with the mapping $ \varphi(x) = (3x, 15x) $.

\paragraph{\tab ii.}

$K - L$ is not a kernel

Let $K(x,y) = \varphi_1(x)\cdot\varphi_1(y)$ and $L(x,y) = \varphi_2(x)\cdot\varphi_2(y)$ \\
Define $\varphi_1(x) = (x^4, x^2)$ and $\varphi_2(x) = (2x^4, -4x^2) $ \\
Notice that $ K(x,y) - L(x,y) = \varphi_1(x)\cdot\varphi_1(y) - \varphi_2(x)\cdot\varphi_2(y) $ \\
$$ = (x^4, x^2)\cdot(y^4,y^2) - (2x^4, -4x^2)\cdot(2y^4, -4y^2) $$ \\
$ = x^4 \cdot y^4 + x^2 \cdot y^2 - 2x^4 \cdot 2y^4 - (-4x^2) \cdot (-4y^2) $ \\
$ = x^4 \cdot y^4 + x^2 \cdot y^2 - 2(x^4 \cdot y^4) - 4(x^2 \cdot y^2) $ \\
$ = -(x^4 \cdot y^4) - 3(x^2 \cdot y^2) $ \\
Notice that this is not a kernel. Proof: \\
ATC that this is a kernel. That means there exists a mapping $\varphi(x)$ \\
s.t. $ -(x^4 \cdot y^4) - 3(x^2 \cdot y^2) = \varphi(x) \cdot \varphi(y) $ \\
Notice that the left side is always negative. \\
However, by the basic properties of an inner product, then $ \varphi(x)\cdot\varphi(x) \geq 0 $ for any $x$. \\
This means we have a norm that is negative, which is a contradiction!
			
\newpage

\section*{\centering{Question 2)}}


\tab Use Lagrange Multipliers to find the maximum and minimum values of the function subject to the given constraints: \\
\tab Function: $f(x,y,z,) = x^2 + y^2 + z^2$.  \\
\tab Constraint: $g(x,y,z) = \frac{x^2}{\alpha^2} + \frac{y^2}{\beta^2} + \frac{z^2}{\beta^2} = 1$, where $ \alpha > \beta > 0$ 


$$
\bigtriangledown f = 
\begin{bmatrix}
\frac{\partial f}{\partial x} \\
\frac{\partial f}{\partial y} \\
\frac{\partial f}{\partial z} 
\end{bmatrix} = 2
\begin{bmatrix}
x \\
y \\
z
\end{bmatrix}
 $$
 
$$
\bigtriangledown g = 
\begin{bmatrix}
\frac{\partial g}{\partial x} \\
\frac{\partial g}{\partial y} \\
\frac{\partial g}{\partial z} 
\end{bmatrix} = 2
\begin{bmatrix}
\frac{x}{\alpha^2} \\
\frac{y}{\beta^2} \\
\frac{z}{\beta^2}
\end{bmatrix}
 $$

$$
\bigtriangledown f  = \lambda\bigtriangledown g
$$

$$
\iff \begin{bmatrix}
x \\
y \\
z
\end{bmatrix}
- \lambda
\begin{bmatrix}
\frac{x}{\alpha^2} \\
\frac{y}{\beta^2} \\
\frac{z}{\beta^2}
\end{bmatrix}
= 0
 $$
 
$$ \iff
\begin{bmatrix}
x(1 - \frac{\lambda}{\alpha^2} ) \\
y(1 - \frac{\lambda}{\beta^2} ) \\
z(1 - \frac{\lambda}{\beta^2} ) \\
\end{bmatrix}
= 0
 $$ \\

Notice that if $ x \neq 0 \Rightarrow \lambda = \alpha^2$ and if $ y \neq 0 \bigvee  z \neq 0 \Rightarrow \lambda = \beta^2$. Therefore:  $\alpha = \beta$ contradicting our constraint.

Consider $x \neq 0$ :

$$ \frac{x^2}{\alpha^2} + \frac{y^2}{\beta^2} + \frac{z^2}{\beta^2} = \frac{x^2}{\alpha^2} = 1 \iff x = \pm \alpha$$

Consider $x = 0$ :

$$ \frac{x^2}{\alpha^2} + \frac{y^2}{\beta^2} + \frac{z^2}{\beta^2} = 
\frac{y^2}{\beta^2} + \frac{z^2}{\beta^2} = 1$$

$$ y^2 + z^2 = \beta^2 $$ \\ 

Notice: $\alpha > \beta > 0 \Rightarrow \alpha^2 > \beta^2$ \\

Therefore we can define the maxima $f(\pm \alpha,0,0) = \alpha^2 $\\

And minima $f(0,x,z) = y^2 + z^2 = \beta^2 $


\section*{\centering{Question 3)}}


\tab Let $X = \mathbb{R}^2$. Let $ C = H =  \lbrace h(a,b,c) = \lbrace(x,y,z) \mid |x| \leq a, |y| \leq b, |z| \leq c \rbrace \mid  a,b,c \in \mathbb{R}^+\rbrace $ the set of all origin centered boxes. \\
\tab Describe the polynomial sample complexity algorithm $L$ that learns $C$ using $H$. State the time complexity and sample complexity of your suggested algorithm. Prove all your steps.


\end{document}
