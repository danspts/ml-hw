\documentclass[12pt]{article}
\usepackage{amsmath}


\linespread{1.3}


\title{Machine Learning from Data: Homework 3 - Probabilities}
\author{227367455 and 323081950}

\begin{document}

\maketitle

\begin{center}
\section*{Question 1}

\end{center}
Given a random sample $\lbrace x_1 , x_2 , ... , x_n \rbrace$, derive the
 maximum likelihood estimator $\hat{p}$ of the Binomial distribution. \\

\begin{center}
$ B(x,p) = \binom{n}{x} p^x (1-p)^{n-x}$ 
\end{center} 

We first want to calculate the likelihood: \\

$ L = P(x_1,...x_n \mid p) =  \prod^{n}_{i = 1} P(x_i \mid p) $ \\

$ =  \prod^{n}_{i = 1} \binom{n}{x_i} p^{x_i} (1-p)^{n-x_i} $ \\

$ = \prod^{n}_{i = 1} p^{x_i} (1-p)^{n-x_i}  \prod^{n}_{i = 1} \binom{n}{x_i}  $ \\

$ = p^{\sum^{n}_{i = 1} x_i} (1-p)^{n^2 - \sum^{n}_{i = 1} x_i}  \prod^{n}_{i = 1} \binom{n}{x_i} $ \\

\newpage

From the likelihood we calculate the log-likelihood: \\

$ ln(L) = ln(p^{\sum^{n}_{i = 1} x_i} (1-p)^{n^2 - \sum^{n}_{i = 1} x_i}  \prod^{n}_{i = 1} \binom{n}{x_i}) $ \\

$ =  ln(p^{\sum^{n}_{i = 1} x_i}) + ln( (1-p)^{n^2 - \sum^{n}_{i = 1} x_i}) + ln(\prod^{n}_{i = 1} \binom{n}{x_i})  $ \\

$ = ln(p) \sum^{n}_{i = 1} x_i + ln( 1-p) (n^2 - \sum^{n}_{i = 1} x_i) +  \sum^{n}_{i = 1} ln(\binom{n}{x_i})$ \\

We will take the derivative in respect to $p$ our given value: \\

$\dfrac{\partial[ln(L) ]}{\partial p} = \dfrac{\partial[ ln(p) \sum^{n}_{i = 1} x_i  ]}{\partial p} + \dfrac{\partial[ ln( 1-p) (n^2 - \sum^{n}_{i = 1} x_i)]}{\partial p} $ \\ 

$ = \dfrac{\sum^{n}_{i = 1} x_i  }{p} - \dfrac{ (n^2 - \sum^{n}_{i = 1} x_i) }{1 - p} $ \\

To find the a maximum we set the derivative to 0 obtaining: \\

$  \dfrac{\sum^{n}_{i = 1} x_i  }{p} - \dfrac{ n^2 - \sum^{n}_{i = 1} x_i }{1 - p} = 0 $ \\

$ (1-p)\sum^{n}_{i = 1} x_i - p (n^2 - \sum^{n}_{i = 1} x_i) = 0 $ \\

$ \sum^{n}_{i = 1} x_i  - p\sum^{n}_{i = 1} x_i - pn^2 + p \sum^{n}_{i = 1} x_i) = 0 $ \\

$ \sum^{n}_{i = 1} x_i  - pn^2 + = 0 $ \\

$ pn^2 = \sum^{n}_{i = 1} x_i  $ \\

Thus we obtain: 

\begin{center}
$ \hat{p} = \dfrac{\sum^{n}_{i = 1} x_i}{n^2} $
\end{center}



\newpage
\section*{Question 2}

A student wants to know her chances to pass and fail an exam if she studies and if she doesn't study.
From last year's results, she sees that
$P(Pass) = 60\%$.
She also found out that 
$P(Studied \mid Pass) = 95\%$, 
$P(Studied \mid Failed) = 60\%$. 
You can assume that every student either studied or didn't study, and either passed or failed.

\subsection*{a.}
What is her probability of passing the exam if she studies?\\

$P(Pass| Studied) = \dfrac{P(Studied \mid Pass)P(Pass)}{P(Studied)}$
\subsection*{b.}

What is her probability of passing if she doesn't study? \\

$P(Pass \mid \overline{Studied}) = \dfrac{P(\overline{Studied} \mid Pass)P(Pass)}{P(\overline{Studied})}$



\newpage
\section*{Question 3}

\subsection*{a.}
\subsection*{b.}
\subsection*{c.}
\subsubsection*{i.}
\subsubsection*{ii.}
\subsubsection*{iii.}

\newpage
\section*{Question 4}

\subsection*{a.} 
\subsection*{b.}
\subsection*{c.}
\subsection*{d.}
    
\end{document}